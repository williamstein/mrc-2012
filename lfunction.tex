\documentclass{article}

\usepackage{amsfonts, amssymb, amsthm, fullpage, amsmath}

\DeclareMathOperator{\Tr}{Tr}


\theoremstyle{plain}
\newtheorem{theorem}{Theorem}
\newtheorem{proposition}{Proposition}

\begin{document}

Let $f$ be a Hilbert modular eigenform of level $\mathfrak{N}$ for $F=\mathbb{Q}(\sqrt{5})$. We write $\mathcal{O}$ for the ring of integers of $F$, $\mathcal{O}_+$ for the totally real elements of $\mathcal{O}$. Let $N$ be a totally positive generator of $\mathfrak{N}$, $d$ be a totally positive generator of the different of $F$ (so $d=\sqrt{5}/\epsilon$ is a good choice), $D$ the discriminant of $F$ (here $D=5$), $\epsilon=\frac{1+\sqrt{5}}{2}$, $\mathfrak{a}$ be an ideal of $\mathcal{O}$. For $a \in F$, we denote by $\bar{a}$ the image of $a$ under the non-trivial field automorphism of $F$, and we write $\mathbb{N}(a)=a\bar{a}$. 

$f$ has a Fourier expansion:
\begin{equation*}
f(z)=\sum_{v \in \mathcal{O}_+} a_v \exp\left(2\pi i\Tr \frac{v z}{d}\right),
\end{equation*}
(the fact that the sum can be restricted to the totally real elements is a theorem) and a completed $L$-function
\begin{align*}
\Lambda(s,f) & = \frac{1}{4\pi^2} \Gamma(s)^2 \mathbb{N}(N)^{s/2} D^s L(s,f)\\
&= \frac{1}{4\pi^2} \Gamma(s)^2 \mathbb{N}(N)^{s/2} D^s \sum_{\mathfrak{a}}a_{\mathfrak{a}}\mathbb{N}(\mathfrak{a})^{-s}.
\end{align*}
Also, because $f$ is an eigenform, it is also an eigenform for the Atkin-Lehner involution $W_N$ with eigenvalue $\epsilon_N= \pm 1$:
\begin{equation*}
f\left(\frac{-1}{Nz_1},\frac{-1}{\bar{N}z_2} \right)=\epsilon_N \mathbb{N}(N)z_1^2z_2^2f(z_1, z_2).
\end{equation*}

\begin{proposition}
We have that:
\begin{equation*}
\Lambda(s,f)= \int_{\mathcal{O}_+^*\backslash \mathbb{R}^2_+} f\left(\frac{iy_1}{\sqrt{N}},\frac{iy_2}{\sqrt{\bar{N}}}\right) y_1^{s-1}y_2^{s-1} dy_1 dy_2
\end{equation*}
\end{proposition}

\begin{proof}
(Here $\epsilon$ is not $\epsilon$ from the beginning, it's just a totally positive unit.) We have
\begin{align*}
\int_{\mathcal{O}_+^*\backslash \mathbb{R}^2_+} f\left(\frac{iy_1}{\sqrt{N}},\frac{iy_2}{\sqrt{\bar{N}}}\right) y_1^{s-1}y_2^{s-1} dy_1 dy_2 &= \int_{\mathcal{O}_+^*\backslash \mathbb{R}^2_+} \sum_{v \in \mathcal{O}_+} a_v \exp\left( -2\pi \left(\frac{vy_1}{\sqrt{N} d}+\frac{\bar{v} y_2}{\sqrt{\bar{N}}\bar{d}}  \right)\right) y_1^{s-1}y_2^{s-1} dy_1 dy_2\\
&= \sum_{\mathfrak{a}} a_{\mathfrak{a}} \int_{\mathcal{O}_+^*\backslash \mathbb{R}^2_+} \sum_{\substack{\mathfrak{a}=(\alpha)\\ \alpha \gg 0 }} \exp\left( -2\pi \left(\frac{\alpha y_1}{\sqrt{N} d}+\frac{\bar{\alpha} y_2}{\sqrt{\bar{N}}\bar{d}}  \right)\right)  y_1^{s-1}y_2^{s-1} dy_1 dy_2\\
& = \sum_{\alpha \in \mathcal{O}_+^* \backslash \mathcal{O}_+} \sum_{\epsilon \in \mathcal{O}_+^*} a_{\alpha \mathcal{O}} \int_{\mathcal{O}_+^*\backslash \mathbb{R}^2_+} \exp\left( -2\pi \left(\frac{\alpha \epsilon y_1}{\sqrt{N} d}+\frac{\bar{\alpha}\bar{\epsilon} y_2}{\sqrt{\bar{N}}\bar{d}}  \right)\right) y_1^{s-1}y_2^{s-1} dy_1 dy_2\\
& = \sum_{\alpha \in \mathcal{O}_+^* \backslash \mathcal{O}_+} a_{\alpha \mathcal{O}} \int_{\mathbb{R}^2_+} \exp\left( -2\pi \left(\frac{\alpha y_1}{\sqrt{N} d}+\frac{\bar{\alpha} y_2}{\sqrt{\bar{N}}\bar{d}}  \right)\right) y_1^{s-1}y_2^{s-1} dy_1 dy_2\\
&= \sum_{\alpha \in \mathcal{O}_+^* \backslash \mathcal{O}_+} a_{\alpha \mathcal{O}} (2 \pi)^{-2s} \left(\frac{\sqrt{N}d}{\alpha}\right)^s\left(\frac{\sqrt{\bar{N}}\bar{d}}{\bar{\alpha}}\right)^s \Gamma(s)^2\\
&= \frac{1}{4\pi^2} \Gamma(s)^2 \mathbb{N}(N)^{s/2} \mathbb{N}(d)^s \sum_{\alpha \in \mathcal{O}_+^* \backslash \mathcal{O}_+} a_{\alpha \mathcal{O}} \mathbb{N}(\alpha)^{-s}  
\end{align*}
Using that  $\mathbb{N}(d)=D$ and $\mathbb{N}(\alpha)= \mathbb{N}(\mathfrak{a})$, this completes the proof.
\end{proof}

A fundamental domain for $\mathcal{O}_+^*\backslash \mathbb{R}^2_+$ is given by $0<y_1$ and $\tau_0 \leq y_2 \leq \epsilon^2 \tau_0$, where $\tau_0$ is any positive real number. Thus
\begin{equation*}
\Lambda(s,f)= \int_{\tau_0}^{\epsilon^2 \tau_0}\int_{0}^{\infty} f\left(\frac{iy_1}{\sqrt{N}},\frac{iy_2}{\sqrt{\bar{N}}}\right) y_1^{s-1}y_2^{s-1} dy_1 dy_2
\end{equation*}

For any positive constant $A$, consider the integral  
\begin{align*}
\int_{\tau_0}^{\epsilon^2 \tau_0}\int_{0}^{A} f\left(\frac{iy_1}{\sqrt{N}},\frac{iy_2}{\sqrt{\bar{N}}}\right) y_1^{s-1}y_2^{s-1} dy_1 dy_2 
\end{align*}
Doing the change of variable $u_1=\frac{1}{y_1}$ $u_2=\frac{1}{y_2}$ and choosing $\tau_0=\frac{1}{\epsilon}$ we have:
\begin{align*}
\int_{\tau_0}^{\epsilon^2 \tau_0}\int_{0}^{A} f\left(\frac{iy_1}{\sqrt{N}},\frac{iy_2}{\sqrt{\bar{N}}}\right) y_1^{s-1}y_2^{s-1} dy_1 dy_2 
&= \int_{\epsilon}^{1/\epsilon}\int_{\infty}^{1/A} f\left(\frac{i}{\sqrt{N}u_1},\frac{i}{\sqrt{\bar{N}}u_2}\right) u_1^{-(s+1)} u_2^{-(s+1)} du_1 du_2\\
&=  \int_{1/\epsilon}^{\epsilon}\int^{\infty}_{1/A} f\left(\frac{i}{\sqrt{N}u_1},\frac{i}{\sqrt{\bar{N}}u_2}\right) u_1^{-(s+1)} u_2^{-(s+1)} du_1 du_2
\end{align*}
But 
\begin{align*}
f\left(\frac{i}{\sqrt{N}u_1},\frac{i}{\sqrt{\bar{N}}u_2}\right)&= \epsilon_N \mathbb{N}(N)\left(\frac{iu_1}{\sqrt{N}}\right)^2\left(\frac{iu_2}{\sqrt{\bar{N}}}\right)^2 f\left(\frac{iu_1}{\sqrt{N}},\frac{iu_2}{\sqrt{\bar{N}}}\right)\\
&= \epsilon_N u_1^2 u_2^2 f\left(\frac{iu_1}{\sqrt{N}},\frac{iu_2}{\sqrt{\bar{N}}}\right)
\end{align*}
So that
\begin{align*}
\int_{1/\epsilon}^{\epsilon}\int_{0}^{A} f\left(\frac{iy_1}{\sqrt{N}},\frac{iy_2}{\sqrt{\bar{N}}}\right) y_1^{s-1}y_2^{s-1}   dy_1 dy_2&=
\epsilon_N \int_{1/\epsilon}^{\epsilon} \int_{1/A}^{\infty} f\left(\frac{iu_1}{\sqrt{N}},\frac{iu_2}{\sqrt{\bar{N}}}\right) u_1^{1-s}u_2^{1-s}du_1 du_2.
\end{align*}
Thus we have, still with our choice of $\tau_0=1/\epsilon$:
\begin{align}\label{niceformula}
\Lambda(s,f)&= \int_{1/\epsilon}^{\epsilon}\int_{A}^{\infty} f\left(\frac{iy_1}{\sqrt{N}},\frac{iy_2}{\sqrt{\bar{N}}}\right) y_1^{s-1}y_2^{s-1} dy_1 dy_2 + \epsilon_N \int_{1/\epsilon}^{\epsilon} \int_{1/A}^{\infty} f\left(\frac{iy_1}{\sqrt{N}},\frac{iy_2}{\sqrt{\bar{N}}}\right) y_1^{1-s}y_2^{1-s} dy_1 dy_2.
\end{align}

\begin{proposition}
There is a formula for $\Lambda(1,s)$ where we can vary a parameter $A$.
\end{proposition}

\begin{proof}
We follow the technique outlined in Demb\'{e}l\'{e}'s paper. Plugging in $s=1$ in Equation \ref{niceformula}, we have
\begin{align*}
\Lambda(1,f)&= \int_{1/\epsilon}^{\epsilon}\int_{A}^{\infty} f\left(\frac{iy_1}{\sqrt{N}},\frac{iy_2}{\sqrt{\bar{N}}}\right) dy_1 dy_2 + \epsilon_N \int_{1/\epsilon}^{\epsilon} \int_{1/A}^{\infty} f\left(\frac{iy_1}{\sqrt{N}},\frac{iy_2}{\sqrt{\bar{N}}}\right) dy_1 dy_2.
\end{align*}

For $C$ any positive real number, we have
\begin{align*}
\int_{C}^{\infty} f\left(\frac{iy_1}{\sqrt{N}},\frac{iy_2}{\sqrt{\bar{N}}}\right) dy_1 &= \sum_{v \in \mathcal{O}_+} a_v  \exp \left( -2\pi \frac{\bar{v} y_2}{\sqrt{\bar{N}}\bar{d}}\right)\int_{C}^{\infty} \exp\left( -2\pi \frac{vy_1}{\sqrt{N} d}\right)  dy_1\\
& =\sum_{v \in \mathcal{O}_+} a_v  \frac{-\sqrt{N} d}{2\pi v} \exp\left( -2\pi \frac{vC}{\sqrt{N} d}\right)   \exp \left( -2\pi \frac{\bar{v} y_2}{\sqrt{\bar{N}}\bar{d}}\right).
\end{align*}
So that
\begin{align*}
\Lambda(1,f) &= \frac{-\sqrt{N} d}{2\pi} \sum_{v \in \mathcal{O}_+} \frac{a_v}{v} \left( \exp\left( -2\pi \frac{vA}{\sqrt{N} d}\right) + \epsilon_N \exp\left( -2\pi \frac{v}{A\sqrt{N} d}\right) \right) \int_{1/\epsilon}^{\epsilon} \exp \left( -2\pi \frac{\bar{v} y_2}{\sqrt{\bar{N}}\bar{d}}\right) dy_2 \\
&= \frac{\mathbb{N}(N)^{1/2} \mathbb{N}(d)}{4\pi^2} \sum_{v \in \mathcal{O}_+} \frac{a_v}{\mathbb{N}(v)} \left( \exp\left( -2\pi \frac{vA}{\sqrt{N} d}\right) + \epsilon_N \exp\left( -2\pi \frac{v}{A\sqrt{N} d}\right) \right) \left( \exp \left( -2\pi \frac{\bar{v} \epsilon}{\sqrt{\bar{N}}\bar{d}}\right) -  \exp \left( -2\pi \frac{\bar{v}}{\epsilon\sqrt{\bar{N}}\bar{d}}\right)\right)
\end{align*}
\end{proof}

We now tackle the derivatives. First, fixing $A=1$ in Equation \ref{niceformula}, we have
\begin{align*}
\Lambda(s,f)&= \int_{1/\epsilon}^{\epsilon}\int_{1}^{\infty} f\left(\frac{iy_1}{\sqrt{N}},\frac{iy_2}{\sqrt{\bar{N}}}\right) (y_1^{s-1}y_2^{s-1} + \epsilon_N y_1^{1-s}y_2^{1-s} )dy_1 dy_2 .
\end{align*}
Differentiating $k$ times with respect to $s$, we get:
\begin{align*}
\Lambda^{(k)}(s,f)&= \int_{1/\epsilon}^{\epsilon}\int_{1}^{\infty} f\left(\frac{iy_1}{\sqrt{N}},\frac{iy_2}{\sqrt{\bar{N}}}\right) (\ln y_1+\ln y_2)^k(y_1^{s-1}y_2^{s-1} +(-1)^k \epsilon_N y_1^{1-s}y_2^{1-s} )dy_1 dy_2 ,
\end{align*}
and evaluating at $s=1$ gives
\begin{align*}
\Lambda^{(k)}(1,f)&= (1 +(-1)^k \epsilon_N) \int_{1/\epsilon}^{\epsilon} \int_{1}^{\infty} f\left(\frac{iy_1}{\sqrt{N}},\frac{iy_2}{\sqrt{\bar{N}}}\right) (\ln y_1+\ln y_2)^kdy_1 dy_2.
\end{align*}

For $k \geq 1$, integrating by parts we have
\begin{align*}
 \int_{1}^{\infty} f\left(\frac{iy_1}{\sqrt{N}},\frac{iy_2}{\sqrt{\bar{N}}}\right) (\ln y_1+\ln y_2)^k dy_1
 &= \sum_{v \in \mathcal{O}_+} a_v  \exp \left( -2\pi \frac{\bar{v} y_2}{\sqrt{\bar{N}}\bar{d}}\right)\int_{1}^{\infty} \exp\left( -2\pi \frac{vy_1}{\sqrt{N} d}\right) (\ln y_1+\ln y_2)^k dy_1\\
 &= \sum_{v \in \mathcal{O}_+} a_v \frac{\sqrt{N}d}{2 \pi v} \exp \left( -2\pi \frac{\bar{v} y_2}{\sqrt{\bar{N}}\bar{d}}\right) \left((\ln y_2)^k \exp\left( -2\pi \frac{v}{\sqrt{N} d}\right) \right. + \\
 & \left. k \int_1^{\infty} \exp\left( -2\pi \frac{vy_1}{\sqrt{N} d}\right) (\ln y_1+\ln y_2)^{k-1} \frac{dy_1}{y_1}\right)\\
% &= \sum_{v \in \mathcal{O}_+} a_v \frac{\sqrt{N}d}{2 \pi v} \exp\left( -2\pi \frac{v}{\sqrt{N} d}\right)\exp \left( -2\pi \frac{\bar{v} y_2}{\sqrt{\bar{N}}\bar{d}}\right) \ln^k y_2  -\\
 %& k \sum_{v \in \mathcal{O}_+} a_v  \exp \left( -2\pi \frac{\bar{v} y_2}{\sqrt{\bar{N}} \bar{d}}\right)  \int_1^{\infty} \exp\left( -2\pi \frac{vy_1}{\sqrt{N} d}\right) (\ln y_1+\ln y_2)^{k-1} \frac{dy_1}{y_1}
\end{align*}

%We now integrate with respect to $y_2$:
%\begin{align*}
%\int_{1/\epsilon}^{\epsilon} \int_{1}^{\infty} f\left(\frac{iy_1}{\sqrt{N}},\frac{iy_2}{\sqrt{\bar{N}}}\right) (\ln y_1+\ln y_2)^kdy_1 dy_2
%&= \sum_{v \in \mathcal{O}_+} a_v \frac{\sqrt{N}d}{2 \pi v} \exp\left( -2\pi \frac{v}{\sqrt{N} d}\right) \int_{1/\epsilon}^{\epsilon}\exp \left( -2\pi \frac{\bar{v} y_2}{\sqrt{\bar{N}}\bar{d}}\right) \ln^k y_2 dy_2 -\\
% & k \sum_{v \in \mathcal{O}_+} a_v  \int_{1/\epsilon}^{\epsilon} \exp \left( -2\pi \frac{\bar{v} y_2}{\sqrt{\bar{N}} \bar{d}}\right)  \int_1^{\infty} \exp\left( -2\pi \frac{vy_1}{\sqrt{N} d}\right) (\ln y_1+\ln y_2)^{k-1} \frac{dy_1}{y_1} dy_2
%\end{align*}

Let
\begin{equation*}
E_r(x)=\int_x^{\infty} \exp(-t) (\ln t)^{r-1}\frac{dt}{t}.
\end{equation*}

When $k=1$, we have:
\begin{align*}
\int_{1}^{\infty} f\left(\frac{iy_1}{\sqrt{N}},\frac{iy_2}{\sqrt{\bar{N}}}\right) (\ln y_1+\ln y_2) dy_1
= & \sum_{v \in \mathcal{O}_+} a_v \frac{\sqrt{N}d}{2 \pi v} \exp\left( -2\pi \frac{v}{\sqrt{N} d}\right)  \exp \left( -2\pi \frac{\bar{v} y_2}{\sqrt{\bar{N}}\bar{d}}\right) \ln y_2 + \\
& \sum_{v \in \mathcal{O}_+} a_v \frac{\sqrt{N}d}{2 \pi v} E_1\left( 2\pi \frac{v}{\sqrt{N} d}\right) \exp \left( -2\pi \frac{\bar{v} y_2}{\sqrt{\bar{N}}\bar{d}}\right)
\end{align*}

We must now integrate with respect to $y_2$. We first do the first integral:
\begin{align*}
\int_{1/\epsilon}^{\epsilon} \exp \left( -2\pi \frac{\bar{v} y_2}{\sqrt{\bar{N}}\bar{d}}\right) \ln y_2 dy_2
 = & -\frac{\sqrt{\bar{N}}\bar{d}}{2 \pi\bar{v}} \left( \exp \left( -2\pi \frac{\bar{v} \epsilon}{\sqrt{\bar{N}}\bar{d}}\right) \ln \epsilon -  \exp \left( -2\pi \frac{\bar{v} }{\sqrt{\bar{N}}\bar{d}\epsilon}\right) \ln(1/\epsilon) \right. \\
 & + \left. E_1\left(2\pi \frac{\bar{v} \epsilon}{\sqrt{\bar{N}}\bar{d}}\right)-E_1\left( 2\pi \frac{\bar{v} }{\sqrt{\bar{N}}\bar{d}\epsilon}\right) \right)\\
= & -\frac{\sqrt{\bar{N}}\bar{d}}{2 \pi\bar{v}} \left( \ln \epsilon \left( \exp \left( -2\pi \frac{\bar{v} \epsilon}{\sqrt{\bar{N}}\bar{d}}\right) +  \exp \left( -2\pi \frac{\bar{v} }{\sqrt{\bar{N}}\bar{d}\epsilon}\right) \right) \right. \\
 & + \left. E_1\left(2\pi \frac{\bar{v} \epsilon}{\sqrt{\bar{N}}\bar{d}}\right)-E_1\left( 2\pi \frac{\bar{v} }{\sqrt{\bar{N}}\bar{d}\epsilon}\right) \right)\\
\end{align*}

Now the second integral:
\begin{align*}
\int_{1/ \epsilon}^{\epsilon} \exp \left( -2\pi \frac{\bar{v} y_2}{\sqrt{\bar{N}}\bar{d}}\right) dy_2 
& =  - \frac{\sqrt{\bar{N}}\bar{d}}{2 \pi \bar{v}} \left( \exp \left( -2\pi \frac{\bar{v} \epsilon}{\sqrt{\bar{N}}\bar{d}}\right) -  \exp \left( -2\pi \frac{\bar{v} }{\sqrt{\bar{N}}\bar{d}\epsilon}\right)\right)
\end{align*}

Altogether we have:
\begin{align*}
\int_{1/\epsilon}^{\epsilon} \int_{1}^{\infty} & f\left(\frac{iy_1}{\sqrt{N}},\frac{iy_2}{\sqrt{\bar{N}}}\right) (\ln y_1+\ln y_2)dy_1 dy_2 \\
= &- \frac{\mathbb{N}(N)^{1/2}\mathbb{N}(d)}{4 \pi^2 }\sum_{v \in \mathcal{O}_+}  \frac{a_v}{\mathbb{N}(v)}  \exp\left( -2\pi \frac{v}{\sqrt{N} d}\right) \left( \ln \epsilon \left( \exp \left( -2\pi \frac{\bar{v} \epsilon}{\sqrt{\bar{N}}\bar{d}}\right) +  \exp \left( -2\pi \frac{\bar{v} }{\sqrt{\bar{N}}\bar{d}\epsilon}\right) \right) \right. \\
& + \left. E_1\left(2\pi \frac{\bar{v} \epsilon}{\sqrt{\bar{N}}\bar{d}}\right)-E_1\left( 2\pi \frac{\bar{v} }{\sqrt{\bar{N}}\bar{d}\epsilon}\right) \right) \\
- & \frac{\mathbb{N}(N)^{1/2}\mathbb{N}(d)}{4 \pi^2 } \sum_{v \in \mathcal{O}_+} \frac{a_v}{\mathbb{N}(v)}  E_1\left( 2\pi \frac{v}{\sqrt{N} d}\right) \left( \exp \left( -2\pi \frac{\bar{v} \epsilon}{\sqrt{\bar{N}}\bar{d}}\right) -  \exp \left( -2\pi \frac{\bar{v} }{\sqrt{\bar{N}}\bar{d}\epsilon}\right)\right)
\end{align*}

\end{document}