\documentclass{article}
\title{Projects for 2012 MRC Arithmetic Statistics Workshop}
\usepackage[margin=1.5in]{geometry}
\usepackage{hyperref}
\include{macros}

\author{William Stein}
\begin{document}
\maketitle
\begin{abstract}
  We give mathematical descriptions of the projects that we will
  attack during the Snowbird workshop.  Two projects involve classical
  modular forms, and two involve Hilbert modular forms for the field
  $\Q(\sqrt{5})$. There is more information about resources for
  sutdying these problems in the wiki
  \url{https://github.com/williamstein/mrc-2012/wiki}.
\end{abstract}

\tableofcontents

\section*{Computing Resources}

\begin{enumerate}
\item I have a cluster with 120 cores and 640GB RAM total at
  University of Washington, which we can use.  

\item I have an account on a supercomputer running Linux with about
  15,000 hours allocated, which we can use.  These use Sun Grid
  Engine, so you submit a job (some script to run), and it runs within
  about 1 day, usually.  (Note that 15,000 hours is actually the same
  as what my cluster can do in 5 days.  But maybe we can use 15,000 hours in one day on the supercomputer.)
\end{enumerate}


\section{Questions about classical modular forms}
\subsection{A question of Barry Mazur about classical modular forms and Hida theory}

This involves gathering arithmetic statistics about a question that
Barry Mazur asked me a week ago, which via some arguments reduced to 
an easy-to-state question about classical modular forms of level one.
Thus this is particularly accessible.


Include other levels?

% goal: 1000; new conjectures

%local rep on local galois barsotti tate group modular curve $X_1(p)$....
%action of inertia group on neron fiber...


\subsection{Structure of singular points of $\Spec(\T)$ for classical higher weight modular forms}

% CM forms at end of Hida 1981 paper.  Computed a little, enough to show interesting.



\section{Questions about Hilbert modular forms}

Henceforth $F=\Q(\sqrt{5})$.

The {\em modularity conjecture} asserts that the set of $L$-functions
$L(E,s)$ attached to elliptic curves over $F$ of conductor an ideal
$\n$ of $F$ is the same as the set $L(f,s)$ of $L$-functions attached
to rational cuspidal Hilbert modular forms of parallel weight $(2,2)$
and level $\n$.



\subsection{The congruence modulus for Hilbert modular forms}

Let $S_{(2,2)}(\n)$ denote the space of Hilbert modular cusp forms of
parallel weight $(2,2)$ and level $\n$.  Let $\T$ be the Hecke algebra
acting on $S_{(2,2)}(\n)$.  

Suppose $M$ is a $\T$-module that is also of
finite rank as a $\Z$-module, and $N\subset M$ is a {\em saturated}
submodule, so $M/N$ is torsion free.  Moreover, suppose that
$M_\Q = M\tensor\Q$ splits as a direct sum $N_{\Q} \oplus N'_{\Q}$,
where $N'_{\Q}$ a $\T$-submodule. 
Define the {\em congruence modulus} of $N$ to be ...



\section{The first elliptic curve over $F$ of  rank 3}
This project is to find the first elliptic curve over $F=\Q(\sqrt{5})$
of rank 3, assuming all standard conjectures about elliptic curves
over $F$, i.e,. assuming modularity and the BSD conjecture.  The
outline is:
\begin{enumerate}
\item Enumerate all rational newforms of norm conductor up to $26569=163^2$ to
sufficient precision that we can compute $\ord_{s=1}(L(f,s))$.
\item Compute $r_f = \ord_{s=1}(L(f,s))$.
\item If there is an $f$ with norm conductor $<26569$ with $\ord_{s=1}(L(f,s))=3$, find a corresponding Weierstrass equation.
\item Collect statistics about the integers $r_f$.
\end{enumerate}
The current rank records are:
\begin{center}
\begin{tabular}{|l|l|l|l|}\hline
Rank & Norm(N) & Equation & Person\\\hline
0 & 31 (prime) &  $[1,a+1,a,a,0]$ &  Dembele \\
1 & 199 (prime) &  $[0,-a-1,1,a,0]$ &  Dembele \\
2 & 1831 (prime) &  $[0,-a,1,-a-1,2a+1]$ & Dembele \\
3 & 26,569$\,=163^2$ &  $[0,0,1,-2,1]$ & Elkies \\
4 & 1,209,079 (prime) & $[1, -1, 0, -8-12a, 19+30a]$ & Elkies \\
5 & 64,004,329 & $[0, -1, 1, -9-2a, 15+4a]$ & Elkies
\\\hline
\end{tabular}
\end{center}


\end{document}
