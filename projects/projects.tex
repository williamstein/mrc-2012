\documentclass{article}
\title{Projects for 2012 MRC Arithmetic Statistics Workshop}
\usepackage[margin=1.5in]{geometry}
\usepackage{hyperref}
\include{macros}

\author{William Stein}
\begin{document}
\maketitle
\begin{abstract}
  We give mathematical descriptions of the projects that we will
  attack during the Snowbird workshop.  Two projects involve classical
  modular forms, and two involve Hilbert modular forms for the field
  $\Q(\sqrt{5})$. There is more information about resources for
  sutdying these problems in the wiki
  \url{https://github.com/williamstein/mrc-2012/wiki}.
\end{abstract}

\tableofcontents

\section*{Background}

Let $\T \subset \End(S)$ be a Hecke algebra, where $S$ is some space
of cusp forms.  Suppose $M$ is a $\T$-module that is also of finite
rank as a $\Z$-module, and $C\subset M$ is a {\em saturated}
submodule, so $M/C$ is torsion free.  Moreover, suppose that $M_\Q =
M\tensor\Q$ splits as a direct sum $C_{\Q} \oplus C'_{\Q}$, where
$C'_{\Q}$ a $\T$-submodule.  Define the the {\em congruence module} of
$C$ to be the $\T$-module $M/(C+C')$ and the {\em congruence number}
of $C$ to be $\#(M/(C +C')) \in \Z$.  

\section{Questions about classical modular forms}

\subsection{A question of Barry Mazur about classical modular forms and Hida theory}

This involves gathering arithmetic statistics about a question that
Barry Mazur asked me a week ago, which via some arguments reduced to 
an easy-to-state question about classical modular forms of level one.
Thus this project is particularly accessible.

For any prime number $p$ and even integer weight $k>2$, let $d(p,k)$
be the number of $p$-adic non-unit eigenvalues of the Hecke operator
$T_p$ acting on the space $S_k(\SL_2(\Z))$ of weight $k$ level $1$
classical modular forms.  For theoretical reasons (Hida theory and an
obvious, but perhaps not yet proven, pattern) we are only interested
in $k\leq (p-3)/2$.  The following is a table of all such
$(p,k)$ with $d(p,k)>0$ for $p<389$:

\begin{center}
\begin{tabular}{|l|c|}\hline
$(p,k)$&{\bf defect}\\\hline
(59,16)&1\\
(79,38)&1\\
(107,28)&1\\
(131,40)&1\\
(139,36)&1\\
(151,60)&1\\
(173,24)&1\\
(193,72)&1\\
(223,72)&1\\
(229,116)&2\\
(257,50)&1\\
\hline\end{tabular}
\begin{tabular}{|l|c|}\hline
$(p,k)$&{\bf defect}\\\hline
(257,100)&1\\
(257,130)&2\\
(263,98)&1\\
(269,78)&1\\
(277,92)&1\\
(283,72)&2\\
(307,78)&1\\
(313,114)&1\\
(331,84)&2\\
(353,76)&2\\
(379,56)&1\\
\hline\end{tabular}
\end{center}

\vspace{1em}
\noindent{\bf The Challenge:} find a way to extend this table up to $p=1024$.
This will provide data for a new conjecture that Barry Mazur intends to make.

%Include other levels?

% goal: 1000; new conjectures

%local rep on local galois barsotti tate group modular curve $X_1(p)$....
%action of inertia group on neron fiber...


\subsection{Congruence for classical higher-weight modular forms}

% CM forms at end of Hida 1981 paper.  Computed a little, enough to show interesting.

Consider $S=S_k(\Gamma_0(N))$ for some $N,k$, and $S(\Z)=S\cap
\Z[[q]]$.  For each of newform $f\in S$ (up to Galois conjugation) of
level dividing $N$, let $C_f \subset S$ be the $\T$-submodule of $S$
spanned by the Galois conjugates of $f$...



\section{Questions about Hilbert modular forms}

Henceforth $F=\Q(\sqrt{5})$.

The {\em modularity conjecture} asserts that the set of $L$-functions
$L(E,s)$ attached to elliptic curves over $F$ of conductor an ideal
$\n$ of $F$ is the same as the set $L(f,s)$ of $L$-functions attached
to rational cuspidal Hilbert modular forms of parallel weight $(2,2)$
and level $\n$.



\subsection{Congruence for Hilbert modular forms}

Let $S_{(2,2)}(\n)$ denote the space of Hilbert modular cusp forms of
parallel weight $(2,2)$ and level $\n$.  Let $\T$ be the Hecke algebra
acting on $S_{(2,2)}(\n)$.  




\section{The first elliptic curve over $F$ of  rank 3}
This project is to find the first elliptic curve over $F=\Q(\sqrt{5})$
of rank 3, assuming all standard conjectures about elliptic curves
over $F$, i.e,. assuming modularity and the BSD conjecture.  The
outline is:
\begin{enumerate}
\item Enumerate all rational newforms of norm conductor up to $26569=163^2$ to
sufficient precision that we can compute $\ord_{s=1}(L(f,s))$.
\item Compute $r_f = \ord_{s=1}(L(f,s))$.
\item If there is an $f$ with norm conductor $<26569$ with $\ord_{s=1}(L(f,s))=3$, find a corresponding Weierstrass equation.
\item Collect statistics about the integers $r_f$.
\end{enumerate}
The current rank records are:
\begin{center}
\begin{tabular}{|l|l|l|l|}\hline
Rank & Norm(N) & Equation & Person\\\hline
0 & 31 (prime) &  $[1,a+1,a,a,0]$ &  Dembele \\
1 & 199 (prime) &  $[0,-a-1,1,a,0]$ &  Dembele \\
2 & 1831 (prime) &  $[0,-a,1,-a-1,2a+1]$ & Dembele \\
3 & 26,569$\,=163^2$ &  $[0,0,1,-2,1]$ & Elkies \\
4 & 1,209,079 (prime) & $[1, -1, 0, -8-12a, 19+30a]$ & Elkies \\
5 & 64,004,329 & $[0, -1, 1, -9-2a, 15+4a]$ & Elkies
\\\hline
\end{tabular}
\end{center}


\end{document}
