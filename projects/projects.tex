\documentclass{article}
\title{Projects for 2012 MRC Arithmetic Statistics Workshop}
\usepackage[margin=1.5in]{geometry}
\usepackage{hyperref}
\include{macros}

\author{William Stein}
\begin{document}
\maketitle
\begin{abstract}
  We give mathematical descriptions of the projects that we will
  attack during the Snowbird workshop.  Two projects involve classical
  modular forms, and two involve Hilbert modular forms for the field
  $\Q(\sqrt{5})$. There is more information about resources for
  sutdying these problems in the wiki
  \url{https://github.com/williamstein/mrc-2012/wiki}.
\end{abstract}

\tableofcontents

\section{Background}

\subsection{Congruence Module and Number}\label{sec:cong}
Let $\T \subset \End(S)$ be a Hecke algebra, where $S$ is some space
of cusp forms.  Suppose $M$ is a $\T$-module that is also of finite
rank as a $\Z$-module, and $C\subset M$ is a {\em saturated}
submodule, so $M/C$ is torsion free.  Moreover, suppose that $M_\Q =
M\tensor\Q$ splits as a direct sum $C_{\Q} \oplus C'_{\Q}$, where
$C'_{\Q}$ a $\T$-submodule.  Define the the {\em congruence module} of
$C$ to be the $\T$-module $M/(C+C')$ and the {\em congruence number}
of $C$ to be $\#(M/(C +C')) \in \Z$.  

\section{Questions about classical modular forms}

\subsection{A question of Barry Mazur about classical modular forms and Hida theory}

This involves gathering arithmetic statistics about a question that
Barry Mazur asked me a week ago, which via some arguments reduced to 
an easy-to-state question about classical modular forms of level one.
Thus this project is particularly accessible.

For any prime number $p$ and even integer weight $k>2$, let $d(p,k)$
be the number of $p$-adic non-unit eigenvalues of the Hecke operator
$T_p$ acting on the space $S_k(\SL_2(\Z))$ of weight $k$ level $1$
classical modular forms.  For theoretical reasons (Hida theory and an
obvious, but perhaps not yet proven, pattern) we are only interested
in $k\leq (p-3)/2$.  The following is a table of all such
$(p,k)$ with $d(p,k)>0$ for $p<389$:

\begin{center}
\begin{tabular}{|l|c|}\hline
$(p,k)$&{\bf defect}\\\hline
(59,16)&1\\
(79,38)&1\\
(107,28)&1\\
(131,40)&1\\
(139,36)&1\\
(151,60)&1\\
(173,24)&1\\
(193,72)&1\\
(223,72)&1\\
(229,116)&2\\
(257,50)&1\\
\hline\end{tabular}
\begin{tabular}{|l|c|}\hline
$(p,k)$&{\bf defect}\\\hline
(257,100)&1\\
(257,130)&2\\
(263,98)&1\\
(269,78)&1\\
(277,92)&1\\
(283,72)&2\\
(307,78)&1\\
(313,114)&1\\
(331,84)&2\\
(353,76)&2\\
(379,56)&1\\
\hline\end{tabular}
\end{center}

\vspace{1em}
\noindent{\bf The Challenge:} find a way to extend this table up to $p=1024$.
This will provide data for a new conjecture that Barry Mazur intends to make.

%Include other levels?

% goal: 1000; new conjectures

%local rep on local galois barsotti tate group modular curve $X_1(p)$....
%action of inertia group on neron fiber...


\subsection{Congruence for classical higher-weight modular forms}

% CM forms at end of Hida 1981 paper.  Computed a little, enough to show interesting.

Consider $S=S_k(\Gamma_0(N))$ for some $N,k$, and $S(\Z)=S\cap
\Z[[q]]$.  For each of newform $f\in S$ (up to Galois conjugation) of
level dividing $N$, let $C_f \subset S$ be the $\T$-submodule of $S$
spanned by the images in $S$ via degeneracy maps of all Galois
conjugates of $f$.  Define the congruence module and number of $f$ as
in Section~\ref{sec:cong} to be the corresponding values for $C_f$ as
a submodule of $S(\Z)$.  Denote by $c_f$ the congruence number.

Given two newforms $f,g \in S$ as above such that $C_f \neq C_g$, define
the congruence module and congruence number of this pair to be
the congruence module and number of $C_f$ in the $\T$-module
$S(\Z)\cap((C_f + C_g)\tensor\Q)$.

Let $\cS = \cS_k(\Gamma_0(N))$, for some $N,k$ be the complex vector
space of cuspidal modular symbols of weight $k$ and level $N$, and let
$S(\Z) = \cS_k(\Gamma_0(N);\Z)$ be the submodule of integral cuspidal
modular symbols.  There is a perfect $\T$-invariant duality $\cS
\times (S \oplus \overline{S}) \to \C$ given by integration, hence for
each newform $f$ and module $C_f$ as above, there is a submodule $D_f
\in S(\Z)$ (note that $\rank_{\Z} D_f = 2\rank_{\Z} C_f$).  Denote by
$d_f$ the congruence number of this submodule.

In the case when $f$ is new of level $N$ has weight $2$, and has
rational Fourier coefficients, I think $\sqrt{d_f}$ is the usual
modular degree of the optimal elliptic curve attached to $f$.
In this case, Ken Ribet proved that $\sqrt{d_f} \mid c_f$, 
and Agashe, Ribet and I formulate and prove a generalization
of this theorem to $f$ with nonrational Fourier coefficients in
\url{http://wstein.org/papers/ars-congruence/}.

When I was a grad student it seems that I computed the first ever
example illustrating that $\sqrt{d_f} \neq c_f$, which was in
$S_2(\Gamma_0(54))$. 


\vspace{1em}
\noindent{\bf The Challenge:} Compute the numbers $c_f$ and $d_f$ for
some forms of weight $\geq 4$.  Is there a relation between
them as is the case for weight $2$?  


It is a theorem of Ken Ribet (that is proved in Mazur's Eisenstein
ideal paper) that if $\T$ is the Hecke algebra associated to
$S_2(\Gamma_0(N))$, then the scheme $\Spec(\T)$ is connected.  At
least when $N$ is prime, I think what this amounts to is that the
following graph is connected: the vertices correspond to Galois orbits
of newforms $f$ and two newform classes $f$ and $g$ are connected by
an edge whenever $f$ and $g$ have congruence number bigger than $1$.
Ribet's argument doesn't seem to generalize to higher weight. 

\vspace{1em}
\noindent{\bf The Challenge:} Compute the graph defined above for
$S_k(\Gamma_0(N))$ for $k\geq 4$ and various $N$.


\section{Questions about Hilbert modular forms}

Henceforth $F=\Q(\sqrt{5})$.

The {\em modularity conjecture} asserts that the set of $L$-functions
$L(E,s)$ attached to elliptic curves over $F$ of conductor an ideal
$\n$ of $F$ is the same as the set $L(f,s)$ of $L$-functions attached
to rational cuspidal Hilbert modular forms of parallel weight $(2,2)$
and level $\n$.



\subsection{Congruence for Hilbert modular forms}

Let $S_{(2,2)}(\n)$ denote the space of Hilbert modular cusp forms of
parallel weight $(2,2)$ and level $\n$.  Let $\T$ be the Hecke algebra
acting on $S_{(2,2)}(\n)$.   Dembele's algorithm involves a finite
rank free $\Z$-module that I'll call $X$ on which $\T$ acts. 
This is the module with basis certain orbits for an action on $\P^1(\cO/\n)$.
Given a newform $f\in S_{(2,2)}(\n)$, there is a corresponding
submodule $C_f \subset X$, and we may consider the congruence module
of $f$ 
and congruence number $c_f$ of $f$ inside $X$.

\vspace{1em}
\noindent{\bf The Challenge:} Compute the number $c_f$ for some Hilbert
newforms $f \in S_{(2,2)}(\n)$.


It is annoying that the definition of $c_f$ above depends on some
perhaps {\em ad hoc} algorithm.  Here is a more intrinsic object that
measures congruences.  The Hecke algebra $\T$ itself is a finite rank
$\Z$-module.  Using Dembele's algorithm we can at least compute the
algebra generated by all $T_{\p}$ with $\p\nmid\n$.  Just as above,
given a newform $f$ there is saturated submodule $D_f$ of $\T$
associated to $f$.    Let $d_f$ be the congruence number of $D_f$.

\vspace{1em}
\noindent{\bf The Challenge:} Compute the number $d_f$ for some
Hilbert newforms $f \in S_{(2,2)}(\n)$.  What is the relation between
$c_f$ and $d_f$?


\section{The first elliptic curve over $F$ of  rank 3}
This project is to find the first elliptic curve over $F=\Q(\sqrt{5})$
of rank 3, assuming all standard conjectures about elliptic curves
over $F$, i.e,. assuming modularity and the BSD conjecture. 
The current rank records are:
\begin{center}
\begin{tabular}{|l|l|l|l|}\hline
Rank & Norm(N) & Equation & Person\\\hline
0 & 31 (prime) &  $[1,a+1,a,a,0]$ &  Dembele \\
1 & 199 (prime) &  $[0,-a-1,1,a,0]$ &  Dembele \\
2 & 1831 (prime) &  $[0,-a,1,-a-1,2a+1]$ & Dembele \\
3 & 26,569$\,=163^2$ &  $[0,0,1,-2,1]$ & Elkies \\
4 & 1,209,079 (prime) & $[1, -1, 0, -8-12a, 19+30a]$ & Elkies \\
5 & 64,004,329 & $[0, -1, 1, -9-2a, 15+4a]$ & Elkies
\\\hline
\end{tabular}
\end{center}

\vspace{1em}
\noindent{\bf The Challenge:} 
Enumerate all {\em rational} newforms of norm conductor up to
$26,569=163^2$ to sufficient precision so that for each we can compute
$r_f = \ord_{s=1}(L(f,s))$.  If there is an $f$ with norm conductor
$<26569$ with $\ord_{s=1}(L(f,s))=3$, find a corresponding Weierstrass
equation.


\end{document}
