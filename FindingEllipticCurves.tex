\documentclass{article}
\usepackage{url}
\usepackage{lmodern} 
\usepackage[parfill]{parskip}
\usepackage{amsmath, amssymb, amsthm} 
\usepackage{mathrsfs} 
\usepackage{graphicx} 
\usepackage[T1]{fontenc} 
\usepackage{mathrsfs} 
\usepackage{textcomp} 
\usepackage{xspace} 
\usepackage{amsthm}

\title{Finding Eigenvectors of Spaces of Hilbert Modular Forms Corresponding to Elliptic Curves}


\newcommand{\Q}{\mathbb{Q}}
\newcommand{\Qbar}{\overline{\Q}}
\newcommand{\Z}{\mathbb{Z}}
\renewcommand{\O}{\mathcal{O}}
\newcommand{\OK}{\mathcal{O}_K}
\newcommand{\OF}{\mathcal{O}_F}
\newcommand{\QF}{\mathbb{Q}(\sqrt{5})}
\newcommand{\fc}{\mathfrak{c}}
\newcommand{\Rc}{R_{\mathfrak{c}}}
\newcommand{\D}{\mathfrak{D}}
\newcommand{\N}{\mathfrak{N}}
\newcommand{\OFN}{\mathcal{O}_{F,\N}}
\newcommand{\OFp}{\mathcal{O}_{F,\p}}
\newcommand{\Fc}{\mathbb{F}_{\mathfrak{c}}}
\newcommand{\p}{\mathfrak{p}}
\newcommand{\Pc}{\mathbb{P}^1(\Fc)}
\newcommand{\Gal}{\text{Gal}}
\newcommand{\hstar}{\mathcal{H}^*}
\newcommand{\iinf}{\iota_{\infty}}
\newtheorem{lem}{Lemma}
\newtheorem{defn}{Definition}
\newtheorem{ex}{Example}
\newtheorem{thm}{Theorem}


\begin{document}
\maketitle

As usual, 
let $F = \QF$ and let $N$ be an ideal of $F$.  
Let $H(N) = S_{\bar{2}}(\Gamma_0(N))$ be the space of Hilbert modular forms of parallel weight two of level $N$.

\section{Immediate Goal} Given a level $N$, 
find all dimension and multiplicity 1 subspaces of $H(N)$ with integral eigenvalues satisfying the Hasse bound.  
I.e., find all the elliptic curves of level $N$.

\section{Motivation}
The motivation is to confirm (or find!) rank records.  
Elkies has shown that the curve $E$ of norm conductor $163^2$ has rank 3.  
When ranked by conductor, is the curve the first curve of rank 3?  
There is a known curve of rank 4, is this the first?  
To check this, 
we need to find all elliptic curves of lower conductor and compute their ranks.  
We approach this by computing appropriate subspaces of $H(N)$ for all $N$ with norm up to $163^2$, 
and hopefully up to ??.

\section{Algorithm}

The algorithm for finding  subspaces of $H(N)$ corresponding to elliptic curves is a two step algorithm.  
First, let $p_1$ be the "smallest" prime coprime to $N$.
We first compute the kernel, $K_a$, of $T_{p_1} - a$ for all $a$ in the Hasse bound, $|a| \leq 2\sqrt{N(p_1)}$.  
Computing these kernels is by far the most expensive operation.  
We explain our speed-up below.
%Sebastian, could you write something about this here?

These kernels are exactly the spaces of $H(N)$ which have $a$ as an eigenvalue.
For each $a$, 
if $K_a$ has dimension 1, 
we've found an elliptic curve and we're done.  
If $K_a$ has dimension 0, 
then there are no abelian varieties with eigenvalue (i.e. $a_{p_1}$) $a$.  
If $K_a$ has dimension greater than 1, 
it corresponds to some combination of integral newforms, integral oldforms, and abelian varieties.  
The dimension of $H(N)$ could be quite large, 
but the $K_a$ should be much smaller.  
If $K_a$ has dimension greater than zero, 
we now assume it is small enough that we can change our method for cutting down this space.
Cutting down  these $K_a$ is the second step.


\subsection{Cutting down subspaces}
Let $K$ be a dimension $d > 1$ subspace of $H(N)$ which has rational eigenvalues for the first few small primes $p_1,...,p_k$.  
$K$ could contain integral oldforms, integral newforms, and abelian varieties, 
the goal is to cut the space $K$ down further so we either find elliptic curves (integral newforms) or we can discard subspaces of $K$.  

We could repeat the first step substituting $K$ for $H(N)$ and computing the kernels of $T_{p_{k+1}} - a$ restricted to $K$ for all $a$ in the Hasse bound of $p_{k+1}$.  
But $K$ is much smaller than $H(N)$ and the norm of $p_{k+1}$ grows as $k$ grows.  
We would be computing many more kernels of a small subspace.  
This isn't going to give us a noticeable speed up, 
thus we might as well just decompose $T_{p_{k+1}}$ with respect to $K$. 
%We can keep decomposing $K$ for $p_{k+1}$ and onward, 
%but how do we know when to stop?

We repeat this decomposition, continuing to cut down.  
The important part is knowing when to throw away subspaces, 
which allows our algorithm to terminate.  
Abelian varieties will have non-integral eigenvalues, 
so will be thrown out when we decompose $T_{p_{k+1}}$ with respect to $K$.  
To check if a subspaces is an integral oldform, 
we check to see if it has $a_p$'s corresponding FINISH THIS!!
%mention decomposition maps and the formula we have
% how we know which spaces could come from old forms.

The following is the second step:

\begin{enumerate}
\item Compute all proper divisors $M$ of $N$.
\item For each $M$, we know the integral newforms and their $a_p$ lists.  
So we compare these lists with the eigenvalues of $K$.  
If they are the same, we can check what the dimension $d_M$ of the old forms from $M$ should be in $H(N)$.  
If $d = d_M$, we're done.
\item Otherwise, we cut down again by taking the hecke matrix $T_{p_{k+1}}$ and decomposing with respect to the subspaces $K$, 
i.e., we find the eigenspaces of $T_{p_{k+1}}$ with eigenvectors $s \in S$.  
This breaks the space $K$ up into the subspaces of $S$ corresponding to particular eigenvalues.
If a subspace $K'$ of $K$ has dimension one and corresponds to an eigenvalue with the Hasse bound, 
we store it and continue.  
If $K'$ has dimension greater than one, 
we go back to $1$ and repeat.
\end{enumerate}

\subsection{Faster methods for computing the kernel of $T_p - a$.}


\section{Long Term Goal}
Find all newforms of a given level and use this to examine rank records for abelian varieties.




\end{document}