\documentclass{article}

\usepackage{amsfonts, amssymb, amsthm, fullpage, amsmath}

\DeclareMathOperator{\Tr}{Tr}


\theoremstyle{plain}
\newtheorem{theorem}{Theorem}
\newtheorem{proposition}{Proposition}
\newtheorem{lemma}{Lemma}

\begin{document}

Let $F$ be a totally real quadratic field of narrow class group $1$, so that every ideal of $F$ has a totally positive generator. Further, let $f$ be a Hilbert modular eigenform of parallel weight $(2,2)$ and level $\mathfrak{N}$ for $F$. We will write $\mathcal{O}$ for the ring of integers of $F$, $\mathcal{O}_+$ for the totally real elements of $\mathcal{O}$, and $\mathfrak{a}$ for an integral ideal of $\mathcal{O}$. Furthermore, let $N$ be a totally positive generator of $\mathfrak{N}$, $d$ be a totally positive generator of the different of $F$, $D$ be the discriminant of $F$ and $\epsilon$ be a fundamental unit of $F$ such that $\epsilon>0$ and $\bar{\epsilon}<0$. (This is always possible since $F$ has narrow class group $1$, so that it has units of every possible sign combination. This implies the existence of such a fundamental unit.) For $a \in F$, we denote by $\bar{a}$ the image of $a$ under the non-trivial field automorphism of $F$, and we write $\mathbb{N}(a)=a\bar{a}$. 

In our particular case, $F=\mathbb{Q}(\sqrt{5})$ so that $D=5$. Some convenient choices for the quantities above are $\epsilon=\frac{1+\sqrt{5}}{2}$, and $d=\sqrt{5}/\epsilon$.

$f$ has a Fourier expansion:
\begin{equation*}
f(z)=\sum_{v \in \mathcal{O}_+} a_v \exp\left(2\pi i\Tr \frac{v z}{d}\right)
\end{equation*}
where
\begin{equation*}
\Tr(\alpha z) = \alpha z_1 + \bar{\alpha}z_2 \qquad \text{for} \quad z=(z_1, z_2) \in \mathbb{H}^2 \quad \text{and} \quad \alpha \in F,
\end{equation*}
(the fact that the sum can be restricted to the totally real elements is a theorem (the Koecher principle), and we do not need $v=0$ since $f$ is a cusp form) and a completed $L$-function
\begin{align*}
\Lambda(s,f) & = \frac{1}{(2\pi)^{2s}} \Gamma(s)^2 \mathbb{N}(N)^{s/2} D^s L(s,f)\\
&= \frac{1}{(2\pi)^{2s}} \Gamma(s)^2 \mathbb{N}(N)^{s/2} D^s \sum_{\mathfrak{a}}a_{\mathfrak{a}}\mathbb{N}(\mathfrak{a})^{-s}.
\end{align*}
(The fact that the Fourier coefficients can be defined on ideals rather than totally positive elements is a lemma, see \cite{bump}.)
%( It follows from the fact that every totally positive unit is the square of a unit, say $\mu = \mu_1^2$, where $\mu_1$ is a unit which is not necessarily totally positive. Then using that $f$ is modular with respect to $\left(\begin{smallmatrix} \mu_1 & 0 \\ 0 & \mu_1^{-1} \end{smallmatrix}\right)$, we get that $f(\mu z)=f(z)$, which implies that $a_{\mu v}= a_{v}$.)
Also, because $f$ is an eigenform, it is also an eigenform for the Atkin-Lehner involution $W_N$ with eigenvalue $\epsilon_N= \pm 1$:
\begin{equation*}
f\left(\frac{-1}{Nz_1},\frac{-1}{\bar{N}z_2} \right)=\epsilon_N \mathbb{N}(N)z_1^2z_2^2f(z_1, z_2).
\end{equation*}

\subsection{Proving and Improving some of Demb\'{e}l\'{e}'s assertions}

\begin{proposition}
We have that:
\begin{equation*}
\Lambda(s,f)= \int_{\mathcal{O}_+^*\backslash \mathbb{R}^2_+} f\left(\frac{iy_1}{\sqrt{N}},\frac{iy_2}{\sqrt{\bar{N}}}\right) y_1^{s-1}y_2^{s-1} dy_1 dy_2
\end{equation*}
\end{proposition}

\begin{proof}
We have
\begin{align*}
\int_{\mathcal{O}_+^*\backslash \mathbb{R}^2_+} f\left(\frac{iy_1}{\sqrt{N}},\frac{iy_2}{\sqrt{\bar{N}}}\right) y_1^{s-1}y_2^{s-1} dy_1 dy_2 &= \int_{\mathcal{O}_+^*\backslash \mathbb{R}^2_+} \sum_{v \in \mathcal{O}_+} a_v \exp\left( -2\pi \left(\frac{vy_1}{\sqrt{N} d}+\frac{\bar{v} y_2}{\sqrt{\bar{N}}\bar{d}}  \right)\right) y_1^{s-1}y_2^{s-1} dy_1 dy_2\\
&= \sum_{\mathfrak{a}} a_{\mathfrak{a}} \int_{\mathcal{O}_+^*\backslash \mathbb{R}^2_+} \sum_{\substack{\mathfrak{a}=(\alpha)\\ \alpha \in \mathcal{O}_+ }} \exp\left( -2\pi \left(\frac{\alpha y_1}{\sqrt{N} d}+\frac{\bar{\alpha} y_2}{\sqrt{\bar{N}}\bar{d}}  \right)\right)  y_1^{s-1}y_2^{s-1} dy_1 dy_2\\
& = \sum_{\alpha \in \mathcal{O}_+^* \backslash \mathcal{O}_+} \sum_{\mu \in \mathcal{O}_+^*} a_{\alpha \mathcal{O}} \int_{\mathcal{O}_+^*\backslash \mathbb{R}^2_+} \exp\left( -2\pi \left(\frac{\alpha \mu y_1}{\sqrt{N} d}+\frac{\bar{\alpha}\bar{\mu} y_2}{\sqrt{\bar{N}}\bar{d}}  \right)\right) y_1^{s-1}y_2^{s-1} dy_1 dy_2\\
& = \sum_{\alpha \in \mathcal{O}_+^* \backslash \mathcal{O}_+} a_{\alpha \mathcal{O}} \int_{\mathbb{R}^2_+} \exp\left( -2\pi \left(\frac{\alpha y_1}{\sqrt{N} d}+\frac{\bar{\alpha} y_2}{\sqrt{\bar{N}}\bar{d}}  \right)\right) y_1^{s-1}y_2^{s-1} dy_1 dy_2\\
&= \sum_{\alpha \in \mathcal{O}_+^* \backslash \mathcal{O}_+} a_{\alpha \mathcal{O}} (2 \pi)^{-2s} \left(\frac{\sqrt{N}d}{\alpha}\right)^s\left(\frac{\sqrt{\bar{N}}\bar{d}}{\bar{\alpha}}\right)^s \Gamma(s)^2\\
&= \frac{1}{(2\pi)^{2s}} \Gamma(s)^2 \mathbb{N}(N)^{s/2} \mathbb{N}(d)^s \sum_{\alpha \in \mathcal{O}_+^* \backslash \mathcal{O}_+} a_{\alpha \mathcal{O}} \mathbb{N}(\alpha)^{-s}  
\end{align*}
Using that  $\mathbb{N}(d)=D$, $\mathbb{N}(\alpha)= \mathbb{N}(\mathfrak{a})$, and that every ideal is generated by a totally positive element, this completes the proof.
\end{proof}

\textbf{Our completed $L$-function is different from Demb\'{e}l\'{e}'s, which is not necessarily obvious at first since he tends to affirm things without proof (hence this whole section). Here is what I know: He defines $L(s,f)$ the same way we do, as the sum $\sum_{\mathfrak{a}}a_{\mathfrak{a}}\mathbb{N}(\mathfrak{a})^{-s}$. But then later on, in the proof of Lemma 2 \cite{dembele}, he writes that by definition
\begin{equation*}
L(f,1)= \int_{\mathcal{O}_+^*\backslash \mathbb{R}^2_+} f\left(iy_1,iy_2\right) dy_1 dy_2.
\end{equation*}
(Notice there are no $\sqrt{N}$'s in the denominator, and he says this is the $L$-function, not the completed $L$-function.) Now I think that he could be lifting this from Bump's book, who only tackles level $1$ and therefore does not have any $\sqrt{N}$'s floating around. In Bump's book this is clearly a formula for the \emph{completed} $L$-function, not the $L$-function, and Demb\'{e}l\'{e} might just not be very careful about the distinction. I get the $\sqrt{N}$'s in the denominators because of the factor of $\mathbb{N}(N)^{s/2}$ in our definition of the completed $L$-function. I personally think that it should be there, but I could never find a reference. I'd like someone else to weigh in on this. Later on, we see that our formula satisfy the functional equation, and I don't think that Demb\'{e}l\'{e}'s does. (I went through the computation once, but I will go through it again to make sure because this stuff gets complicated and I get bored and I make mistakes.)}

A fundamental domain for $\mathcal{O}_+^*\backslash \mathbb{R}^2_+$ is given by $0<y_1$ and $\tau_0 \leq y_2 < \epsilon^2 \tau_0$, where $\tau_0$ is any positive real number. Thus
\begin{equation*}
\Lambda(s,f)= \int_{\tau_0}^{\epsilon^2 \tau_0}\int_{0}^{\infty} f\left(\frac{iy_1}{\sqrt{N}},\frac{iy_2}{\sqrt{\bar{N}}}\right) y_1^{s-1}y_2^{s-1} dy_1 dy_2.
\end{equation*}

For any positive constant $A$, consider the integral  
\begin{align*}
\int_{\tau_0}^{\epsilon^2 \tau_0}\int_{0}^{A} f\left(\frac{iy_1}{\sqrt{N}},\frac{iy_2}{\sqrt{\bar{N}}}\right) y_1^{s-1}y_2^{s-1} dy_1 dy_2 .
\end{align*}
Doing the change of variable $u_1=\frac{1}{y_1}$ $u_2=\frac{1}{y_2}$ and choosing $\tau_0=\frac{1}{\epsilon}$ we have:
\begin{align*}
\int_{\tau_0}^{\epsilon^2 \tau_0}\int_{0}^{A} f\left(\frac{iy_1}{\sqrt{N}},\frac{iy_2}{\sqrt{\bar{N}}}\right) y_1^{s-1}y_2^{s-1} dy_1 dy_2 
&= \int_{\epsilon}^{1/\epsilon}\int_{\infty}^{1/A} f\left(\frac{i}{\sqrt{N}u_1},\frac{i}{\sqrt{\bar{N}}u_2}\right) u_1^{-(s+1)} u_2^{-(s+1)} du_1 du_2\\
&=  \int_{1/\epsilon}^{\epsilon}\int^{\infty}_{1/A} f\left(\frac{i}{\sqrt{N}u_1},\frac{i}{\sqrt{\bar{N}}u_2}\right) u_1^{-(s+1)} u_2^{-(s+1)} du_1 du_2.
\end{align*}
But 
\begin{align*}
f\left(\frac{i}{\sqrt{N}u_1},\frac{i}{\sqrt{\bar{N}}u_2}\right)&= \epsilon_N \mathbb{N}(N)\left(\frac{iu_1}{\sqrt{N}}\right)^2\left(\frac{iu_2}{\sqrt{\bar{N}}}\right)^2 f\left(\frac{iu_1}{\sqrt{N}},\frac{iu_2}{\sqrt{\bar{N}}}\right)\\
&= \epsilon_N u_1^2 u_2^2 f\left(\frac{iu_1}{\sqrt{N}},\frac{iu_2}{\sqrt{\bar{N}}}\right).
\end{align*}
So that
\begin{align*}
\int_{1/\epsilon}^{\epsilon}\int_{0}^{A} f\left(\frac{iy_1}{\sqrt{N}},\frac{iy_2}{\sqrt{\bar{N}}}\right) y_1^{s-1}y_2^{s-1}   dy_1 dy_2&=
\epsilon_N \int_{1/\epsilon}^{\epsilon} \int_{1/A}^{\infty} f\left(\frac{iu_1}{\sqrt{N}},\frac{iu_2}{\sqrt{\bar{N}}}\right) u_1^{1-s}u_2^{1-s}du_1 du_2.
\end{align*}
Thus we have, still with our choice of $\tau_0=1/\epsilon$:
\begin{align}\label{niceformula}
\Lambda(s,f)&= \int_{1/\epsilon}^{\epsilon}\int_{A}^{\infty} f\left(\frac{iy_1}{\sqrt{N}},\frac{iy_2}{\sqrt{\bar{N}}}\right) y_1^{s-1}y_2^{s-1} dy_1 dy_2 + \epsilon_N \int_{1/\epsilon}^{\epsilon} \int_{1/A}^{\infty} f\left(\frac{iy_1}{\sqrt{N}},\frac{iy_2}{\sqrt{\bar{N}}}\right) y_1^{1-s}y_2^{1-s} dy_1 dy_2.
\end{align}
And with $A=1$, we have:
\begin{align}
\Lambda(s,f)&= \int_{1/\epsilon}^{\epsilon}\int_{1}^{\infty} f\left(\frac{iy_1}{\sqrt{N}},\frac{iy_2}{\sqrt{\bar{N}}}\right) y_1^{s-1}y_2^{s-1} dy_1 dy_2 + \epsilon_N \int_{1/\epsilon}^{\epsilon} \int_{1}^{\infty} f\left(\frac{iy_1}{\sqrt{N}},\frac{iy_2}{\sqrt{\bar{N}}}\right) y_1^{1-s}y_2^{1-s} dy_1 dy_2,
\end{align}
from which one can see that $\Lambda(s,f)=\epsilon_N \Lambda(2-s,f)$, so that our completed $L$-function satisfies a functional equation.

Recall the incomplete gamma function, for $\mathfrak{R}(x)>0$ and $s \in \mathbb{C}$:
\begin{equation*}
\Gamma(s, x)=x^s\int_1^{\infty}e^{-xt}t^s \frac{dt}{t}
\end{equation*}

\begin{proposition}
There is a formula for $\Lambda(s,f)$ where we can vary a parameter $A$, where $A$ is a positive real number.
\end{proposition}

\begin{proof}
We follow a hybrid of the technique outlined in Demb\'{e}l\'{e}'s paper \cite{dembele} and that in Cohen's book \cite{cohen}.
Consider the integral
\begin{equation}\label{firstintegral}
\begin{split}
\int_{1/\epsilon}^{\epsilon}\int_{A}^{\infty} &f\left(\frac{iy_1}{\sqrt{N}},\frac{iy_2}{\sqrt{\bar{N}}}\right) y_1^{s-1}y_2^{s-1} dy_1 dy_2 \\
&= \sum_{v \in \mathcal{O}_+} a_v  \int_{1/\epsilon}^{\epsilon} \exp \left( \frac{ -2\pi\bar{v} y_2}{\sqrt{\bar{N}}\bar{d}}\right) y_2^{s-1} dy_2 \int_{A}^{\infty} \exp\left(  \frac{-2\pi vy_1}{\sqrt{N} d}\right) y_1^{s-1} dy_1\\
&= \left(\frac{\mathbb{N}(N)^{1/2}D}{4\pi^2}\right)^s \sum_{v \in \mathcal{O}_+} \frac{a_v}{\mathbb{N}(v)^s} \Gamma\left(s, \frac{2\pi v A}{\sqrt{N}d}\right) \left(\Gamma\left(s,\frac{2 \pi \bar{v}}{\sqrt{\bar{N}}\bar{d} \epsilon} \right)-  \Gamma\left(s,\frac{2 \pi \bar{v}\epsilon}{\sqrt{\bar{N}}\bar{d} } \right) \right) 
\end{split}
\end{equation}
Similarly,
\begin{equation}\label{secondintegral}
\begin{split}
\int_{1/\epsilon}^{\epsilon}\int_{1/A}^{\infty} &f\left(\frac{iy_1}{\sqrt{N}},\frac{iy_2}{\sqrt{\bar{N}}}\right) y_1^{1-s}y_2^{1-s} dy_1 dy_2 \\
&= \sum_{v \in \mathcal{O}_+} a_v  \int_{1/\epsilon}^{\epsilon} \exp \left( \frac{ -2\pi\bar{v} y_2}{\sqrt{\bar{N}}\bar{d}}\right) y_2^{1-s} dy_2 \int_{1/A}^{\infty} \exp\left(  \frac{-2\pi vy_1}{\sqrt{N} d}\right) y_1^{1-s} dy_1\\
&= \left(\frac{\mathbb{N}(N)^{1/2}D}{4\pi^2}\right)^{2-s}\sum_{v \in \mathcal{O}_+} \frac{a_v}{\mathbb{N}(v)^{2-s}} \Gamma\left(2-s, \frac{2\pi v }{A\sqrt{N}d}\right) \left(\Gamma\left(2-s,\frac{2 \pi \bar{v}}{\sqrt{\bar{N}}\bar{d} \epsilon} \right)-  \Gamma\left(2-s,\frac{2 \pi \bar{v}\epsilon}{\sqrt{\bar{N}}\bar{d} } \right) \right) 
\end{split}
\end{equation}
Now adding the result of Equation \ref{firstintegral} and $\epsilon_N$ times the result of Equation \ref{secondintegral} gives the formula whose existence is asserted.
\end{proof}

In particular, when $s=1$, we get:
\begin{equation*}
\begin{split}
\Lambda(1,f)= & \frac{\mathbb{N}(N)^{1/2}\mathbb{N}(d)}{4\pi^2} \sum_{v \in \mathcal{O}_+} \frac{a_v}{\mathbb{N}(v)} \Gamma\left(1, \frac{2\pi v A}{\sqrt{N}d}\right) \left(\Gamma\left(1,\frac{2 \pi \bar{v}}{\sqrt{\bar{N}}\bar{d} \epsilon} \right)-  \Gamma\left(1,\frac{2 \pi \bar{v}\epsilon}{\sqrt{\bar{N}}\bar{d} } \right) \right) \\
&+  \epsilon_N \frac{\mathbb{N}(N)^{1/2}\mathbb{N}(d)}{4\pi^2} \sum_{v \in \mathcal{O}_+} \frac{a_v}{\mathbb{N}(v)} \Gamma\left(1, \frac{2\pi v }{A\sqrt{N}d}\right) \left(\Gamma\left(1,\frac{2 \pi \bar{v}}{\sqrt{\bar{N}}\bar{d} \epsilon} \right)-  \Gamma\left(1,\frac{2 \pi \bar{v}\epsilon}{\sqrt{\bar{N}}\bar{d} } \right) \right)\\
= &  \frac{\mathbb{N}(N)^{1/2}\mathbb{N}(d)}{4\pi^2} \sum_{v \in \mathcal{O}_+} \frac{a_v}{\mathbb{N}(v)} \left( \Gamma\left(1, \frac{2\pi v A}{\sqrt{N}d}\right) +\epsilon_N\Gamma\left(1, \frac{2\pi v }{A\sqrt{N}d}\right) \right)\left(\Gamma\left(1,\frac{2 \pi \bar{v}}{\sqrt{\bar{N}}\bar{d} \epsilon} \right)-  \Gamma\left(1,\frac{2 \pi \bar{v}\epsilon}{\sqrt{\bar{N}}\bar{d} } \right) \right)
\end{split}
\end{equation*}
Since $\Gamma(1,x)=e^{-x}$, this ``simplifies" to:
\begin{align*}
\Lambda(1,f) & = \frac{\mathbb{N}(N)^{1/2} \mathbb{N}(d)}{4\pi^2} \sum_{v \in \mathcal{O}_+} \frac{a_v}{\mathbb{N}(v)} \left( \exp\left(\frac{-2\pi vA}{\sqrt{N} d}\right) + \epsilon_N \exp\left(\frac{-2\pi v}{A\sqrt{N} d}\right) \right) \left( \exp \left(\frac{-2\pi \bar{v} \epsilon}{\sqrt{\bar{N}}\bar{d}}\right) -  \exp \left(\frac{-2\pi \bar{v}}{\epsilon\sqrt{\bar{N}}\bar{d}}\right)\right)\\
& = \frac{\mathbb{N}(N)^{1/2} D}{4\pi^2} \sum_{v \in \mathcal{O}_+} \frac{a_v}{\mathbb{N}(v)} \left( \exp\left(\frac{-2\pi vA\epsilon}{\sqrt{ND} }\right) + \epsilon_N \exp\left(\frac{-2\pi v\epsilon}{A\sqrt{ND}}\right) \right) \left( \exp \left(\frac{-2\pi \bar{v} }{\sqrt{\bar{N}D}}\right) -  \exp \left(\frac{-2\pi \bar{v}}{\epsilon^2\sqrt{\bar{N}D}}\right)\right)
\end{align*}
This formula can be used to compute the sign of the functional equation for $\Lambda(s,f)$. (The second expression has $D$ in it rather than $d$, since this is likely what we would use in practice. The very last exponential appearing in the expression (the one with $\epsilon^2$ in it) uses the fact that we are working with $F=\mathbb{Q}(\sqrt{5})$ and is not valid for a general real quadratic number field of narrow class number 1.)

Our result agrees with the formula presented in Lemma 2 of \cite{dembele}, up to the factor of $\mathbb{N}(N)^{1/2}$, which was discussed earlier. We take this opportunity to note that in his paper, Demb\'{e}l\'{e} does not specify that this formula is only valid for $F=\mathbb{Q}(\sqrt{5})$, since it crucially uses that $\epsilon^2=\epsilon+1$. (He does have the assumption in his thesis.)

\textbf{Kind of annoyingly, Demb\'{e}l\'{e} computes in fact
\begin{equation*}
\Lambda(s,f)= \int_{\tau_0}^{\epsilon^2 \tau_0}\int_{0}^{\infty} f\left(iy_1,iy_2\right)  dy_2 dy_1.
\end{equation*}
which explains why our formula is slight off from his. But either domains are a fundamental domain for the space $\mathcal{O}_+^*\backslash \mathbb{R}^2_+$. His formula is a bit cleaner to write, but I think that computationally they should be the same.}

%Plugging in $A=1$ in this last equation, we have, when $F=\mathbb{Q}(\sqrt{5})$:
%\begin{align*}
%\Lambda(1,f)&=(1+\epsilon_N) \frac{\mathbb{N}(N)^{1/2} D}{4\pi^2} \sum_{v \in \mathcal{O}_+} \frac{a_v}{\mathbb{N}(v)}  \exp\left(\frac{-2\pi v\epsilon}{\sqrt{ND} }\right) \left( \exp \left(\frac{-2\pi \bar{v} }{\sqrt{\bar{N}D}}\right) -  \exp \left(\frac{-2\pi \bar{v}}{\epsilon^2\sqrt{\bar{N}D}}\right)\right)\\
%&=(1+\epsilon_N) \frac{\mathbb{N}(N)^{1/2} D}{4\pi^2} \sum_{v \in \mathcal{O}_+} \frac{a_v}{\mathbb{N}(v)}  \exp\left(\frac{2 \pi}{\sqrt{D}} \left(\frac{\bar{v}\bar{\epsilon}}{\sqrt{\bar{N}}}-\frac{v}{\sqrt{N}}\right) \right) \left( 1 - \exp \left( \frac{-2\pi v \epsilon}{\sqrt{ND}}\right) \right)
%\end{align*}





\subsection{Now the derivatives}

Choosing $A=1$, we have:
\begin{equation*}
\begin{split}
\Lambda(s,f)=& \left(\frac{\mathbb{N}(N)^{1/2}D}{4\pi^2}\right)^s \sum_{v \in \mathcal{O}_+} \frac{a_v}{\mathbb{N}(v)^s} \Gamma\left(s, \frac{2\pi v }{\sqrt{N}d}\right) \left(\Gamma\left(s,\frac{2 \pi \bar{v}}{\sqrt{\bar{N}}\bar{d} \epsilon} \right)-  \Gamma\left(s,\frac{2 \pi \bar{v}\epsilon}{\sqrt{\bar{N}}\bar{d} } \right) \right) \\
&+ \left(\frac{\mathbb{N}(N)^{1/2}D}{4\pi^2}\right)^{2-s}\sum_{v \in \mathcal{O}_+} \frac{a_v}{\mathbb{N}(v)^{2-s}} \Gamma\left(2-s, \frac{2\pi v }{\sqrt{N}d}\right) \left(\Gamma\left(2-s,\frac{2 \pi \bar{v}}{\sqrt{\bar{N}}\bar{d} \epsilon} \right)-  \Gamma\left(2-s,\frac{2 \pi \bar{v}\epsilon}{\sqrt{\bar{N}}\bar{d} } \right) \right)
\end{split}
\end{equation*}

We must now differentiate. Following Cohen \cite{cohen}, we define the following functions by induction:
\begin{equation*}
\Gamma_{-1}(s,x)= e^{-x}x^s \qquad \text{and} \qquad \Gamma_r(s,x) = \int_x^{\infty}\frac{\Gamma_{r-1}(s,t)}{t}dt \quad \text{for } r\geq 0.
\end{equation*}
In particular, $\Gamma_0(s,x)=\Gamma(s,x)$ from before and $\Gamma_1(1,x)=E_1(x)$, the exponential integral.

Cohen shows the following:
\begin{lemma}
For $r \geq 0$ we have:
\begin{equation*}
\Gamma_r(s,x)=x^s \int_1^{\infty} \frac{(\ln t)^r}{r!}e^{-xt} t^s \frac{dt}{t}=\int_x^{\infty} \frac{(\ln(t/x))^r}{r!}e^{-t}t^s \frac{dt}{t}.
\end{equation*}
\end{lemma}

As a consequence, 
\begin{equation*}
\frac{d}{ds} \Gamma_r(s,x)= \Gamma_r(s,x) \ln x +(r+1)\Gamma_{r+1}(s,x).
\end{equation*}

We need some lemmas:
\begin{lemma}
Let $A$, $B$ and $C$ be positive real numbers and $r_1$ and $r_2$ be two positive integers. Then
\begin{equation*}
\begin{split}
\frac{d^k}{ds^k} & \left( A^{s} \Gamma_{r_1}(s,B) \Gamma_{r_2}(s, C)\right)\\
&= \sum_{i=0}^{k}\sum_{j=0}^{i}\binom{k}{i}\binom{i}{j} (\ln ABC)^{k-i} \prod_{l_1=1}^{j} (r_1+l_1) \prod_{l_2=1}^{i-j}(r_2+l_2)
A^{s} \Gamma_{r_1+j}(s,B) \Gamma_{r_2+i-j}(s, C)
\end{split}
\end{equation*}
where $\prod_{j=1}^{0} x =1$.
\end{lemma}

\begin{proof}
Since $A$, $B$ and $C$ are fixed, we write $F(s, r_1, r_2)=A^{s} \Gamma_{r_1}(s,B) \Gamma_{r_2}(s, C)$. Then the Lemma reads:
\begin{equation*}
\frac{d^k}{ds^k}   F(s, r_1,r_2) = \sum_{i=0}^{k}\sum_{j=0}^{i}\binom{k}{i}\binom{i}{j} (\ln ABC)^{k-i} \prod_{l_1=1}^{j} (r_1+l_1) \prod_{l_2=1}^{i-j}(r_2+l_2)F(s, r_1+j,r_2+i-j)
\end{equation*}

We prove this by induction. For the case $k=1$ we have:
\begin{equation*}
\frac{d}{ds}  F(s, r_1,r_2) =\ln ABC \, F(s, r_1,r_2) + (r_1+1) F(s, r_1+1,r_2)+(r_2+1)F(s, r_1,r_2+1),
\end{equation*}
as claimed.

Assuming the formula for $k$, we now prove it for $k+1$:
\begin{align*}
\frac{d^{k+1}}{ds^{k+1}} F(s, r_1,r_2) = & \sum_{i=0}^{k}\sum_{j=0}^{i}\binom{k}{i}\binom{i}{j} (\ln ABC)^{k-i} \prod_{l_1=1}^{j} (r_1+l_1) \prod_{l_2=1}^{i-j}(r_2+l_2)\frac{d}{ds}F(s, r_1+j,r_2+i-j)\\
= & \sum_{i=0}^{k}\sum_{j=0}^{i}\binom{k}{i}\binom{i}{j} (\ln ABC)^{k-i+1} \prod_{l_1=1}^{j} (r_1+l_1) \prod_{l_2=1}^{i-j}(r_2+l_2)  F(s, r_1+j,r_2+i-j) \\
& + \sum_{i=0}^{k}\sum_{j=0}^{i}\binom{k}{i}\binom{i}{j} (\ln ABC)^{k-i} \prod_{l_1=1}^{j+1} (r_1+l_1) \prod_{l_2=1}^{i-j}(r_2+l_2) F(s, r_1+j+1,r_2+i-j) \\
& + \sum_{i=0}^{k}\sum_{j=0}^{i}\binom{k}{i}\binom{i}{j} (\ln ABC)^{k-i} \prod_{l_1=1}^{j} (r_1+l_1) \prod_{l_2=1}^{i-j+1}(r_2+l_2) F(s, r_1+j,r_2+i-j+1)\\
\end{align*}

We fix a pair $(m,n)$, and gather the coefficient of the term $F(s, r_1+m, r_2+n-m)$ in this sum:
\begin{equation*}
\begin{split}
(\ln ABC)^{k-n+1} & \prod_{l_1=1}^{m} (r_1+l_1) \prod_{l_2=1}^{n-m}(r_2+l_2) \left( \binom{k}{n}\binom{n}{m} + \binom{k}{n-1}\binom{n-1}{m-1}+ \binom{k}{n-1}\binom{n-1}{m} \right) \\
& = (\ln ABC)^{k-n+1}  \prod_{l_1=1}^{m} (r_1+l_1) \prod_{l_2=1}^{n-m}(r_2+l_2) \binom{k+1}{n} \binom{n}{m}
\end{split}
\end{equation*}

\end{proof}

To get the derivatives of $\Lambda(s,f)$ we could take the derivative of this, like Cohen does, but it seems painful. Instead, we will use Equation \ref{niceformula}.

We now tackle the derivatives. First, fixing $A=1$ in Equation \ref{niceformula}, we have
\begin{equation*}
\Lambda(s,f)= \int_{1/\epsilon}^{\epsilon}\int_{1}^{\infty} f\left(\frac{iy_1}{\sqrt{N}},\frac{iy_2}{\sqrt{\bar{N}}}\right) (y_1^{s-1}y_2^{s-1} + \epsilon_N y_1^{1-s}y_2^{1-s} )dy_1 dy_2 .
\end{equation*}
Differentiating $k$ times with respect to $s$, we get:
\begin{equation*}
\Lambda^{(k)}(s,f)= \int_{1/\epsilon}^{\epsilon}\int_{1}^{\infty} f\left(\frac{iy_1}{\sqrt{N}},\frac{iy_2}{\sqrt{\bar{N}}}\right) (\ln y_1+\ln y_2)^k(y_1^{s-1}y_2^{s-1} +(-1)^k \epsilon_N y_1^{1-s}y_2^{1-s} )dy_1 dy_2 ,
\end{equation*}
and evaluating at $s=1$ gives
\begin{align*}
\Lambda^{(k)}(1,f)&=  (1 +(-1)^k \epsilon_N) \int_{1/\epsilon}^{\epsilon} \int_{1}^{\infty} f\left(\frac{iy_1}{\sqrt{N}},\frac{iy_2}{\sqrt{\bar{N}}}\right) (\ln y_1+\ln y_2)^kdy_1 dy_2\\
&= (1 +(-1)^k \epsilon_N) \int_{1/\epsilon}^{\epsilon} \int_{1}^{\infty} f\left(\frac{iy_1}{\sqrt{N}},\frac{iy_2}{\sqrt{\bar{N}}}\right) \left( \sum_{i=0}^{k}\binom{k}{i}(\ln y_1)^i (\ln y_2)^{k-i} \right) dy_1 dy_2\\
&= (1 +(-1)^k \epsilon_N) \sum_{v \in \mathcal{O}_+} a_v \sum_{i=0}^{k}\binom{k}{i} \int_{1/\epsilon}^{\epsilon} \exp \left( \frac{ -2\pi\bar{v} y_2}{\sqrt{\bar{N}}\bar{d}}\right) (\ln y_2)^{k-i} dy_2  \int_{1}^{\infty} \exp\left(  \frac{-2\pi vy_1}{\sqrt{N} d}\right) (\ln y_1)^i  dy_1\\
\end{align*}

For $k \geq 1$, integrating by parts we have
\begin{equation*}
\begin{split}
\int_{1/\epsilon}^{\epsilon} \int_{1}^{\infty} & f\left(\frac{iy_1}{\sqrt{N}},\frac{iy_2}{\sqrt{\bar{N}}}\right) (\ln y_1+\ln y_2)^k dy_1dy_2\\
 =& \sum_{v \in \mathcal{O}_+} a_v  \int_{1/\epsilon}^{\epsilon} \exp \left(\frac{ -2\pi \bar{v} y_2}{\sqrt{\bar{N}}\bar{d}}\right)\int_{1}^{\infty} \exp\left( \frac{ -2\pi  vy_1}{\sqrt{N} d}\right) (\ln y_1+\ln y_2)^k dy_1dy_2\\
 =& \sum_{v \in \mathcal{O}_+} a_v \frac{\sqrt{N}d}{2 \pi v} \int_{1/\epsilon}^{\epsilon} \exp \left( \frac{ -2\pi\bar{v} y_2}{\sqrt{\bar{N}}\bar{d}}\right) \left((\ln y_2)^k \exp\left( \frac{ -2\pi v}{\sqrt{N} d}\right)  + k \int_1^{\infty} \exp\left(\frac{ -2\pi vy_1}{\sqrt{N} d}\right) (\ln y_1+\ln y_2)^{k-1} \frac{dy_1}{y_1}\right) dy_2\\
 =& \sum_{v \in \mathcal{O}_+} a_v \frac{\sqrt{N}d}{2 \pi v} \exp\left(\frac{ -2\pi  v}{\sqrt{N} d}\right) \int_{1/\epsilon}^{\epsilon}\exp \left(\frac{ -2\pi \bar{v} y_2}{\sqrt{\bar{N}}\bar{d}}\right) (\ln y_2)^k dy_2\\
  & -k \sum_{v \in \mathcal{O}_+} a_v  \exp \left(\frac{ -2\pi \bar{v} y_2}{\sqrt{\bar{N}} \bar{d}}\right)  \int_1^{\infty} \exp\left(\frac{ -2\pi  vy_1}{\sqrt{N} d}\right) (\ln y_1+\ln y_2)^{k-1} \frac{dy_1}{y_1}
\end{split}
\end{equation*}

%We now integrate with respect to $y_2$:
%\begin{align*}
%\int_{1/\epsilon}^{\epsilon} \int_{1}^{\infty} f\left(\frac{iy_1}{\sqrt{N}},\frac{iy_2}{\sqrt{\bar{N}}}\right) (\ln y_1+\ln y_2)^kdy_1 dy_2
%&= \sum_{v \in \mathcal{O}_+} a_v \frac{\sqrt{N}d}{2 \pi v} \exp\left( -2\pi \frac{v}{\sqrt{N} d}\right) \int_{1/\epsilon}^{\epsilon}\exp \left( -2\pi \frac{\bar{v} y_2}{\sqrt{\bar{N}}\bar{d}}\right) \ln^k y_2 dy_2 -\\
% & k \sum_{v \in \mathcal{O}_+} a_v  \int_{1/\epsilon}^{\epsilon} \exp \left( -2\pi \frac{\bar{v} y_2}{\sqrt{\bar{N}} \bar{d}}\right)  \int_1^{\infty} \exp\left( -2\pi \frac{vy_1}{\sqrt{N} d}\right) (\ln y_1+\ln y_2)^{k-1} \frac{dy_1}{y_1} dy_2
%\end{align*}

Let
\begin{equation*}
G_r(x)=\frac{1}{(r-1)!}\int_1^{\infty} \exp(-xt) (\ln t)^{r-1}\frac{dt}{t}.
\end{equation*}

When $k=1$, we have:
\begin{align*}
\int_{1}^{\infty} f\left(\frac{iy_1}{\sqrt{N}},\frac{iy_2}{\sqrt{\bar{N}}}\right) (\ln y_1+\ln y_2) dy_1
= & \sum_{v \in \mathcal{O}_+} a_v \frac{\sqrt{N}d}{2 \pi v} \exp\left( -2\pi \frac{v}{\sqrt{N} d}\right)  \exp \left( -2\pi \frac{\bar{v} y_2}{\sqrt{\bar{N}}\bar{d}}\right) \ln y_2 + \\
& \sum_{v \in \mathcal{O}_+} a_v \frac{\sqrt{N}d}{2 \pi v} G_1\left( 2\pi \frac{v}{\sqrt{N} d}\right) \exp \left( -2\pi \frac{\bar{v} y_2}{\sqrt{\bar{N}}\bar{d}}\right)
\end{align*}

We must now integrate with respect to $y_2$. We first do the first integral:
\begin{align*}
\int_{1/\epsilon}^{\epsilon} \exp \left( -2\pi \frac{\bar{v} y_2}{\sqrt{\bar{N}}\bar{d}}\right) \ln y_2 dy_2
 = & -\frac{\sqrt{\bar{N}}\bar{d}}{2 \pi\bar{v}} \left( \exp \left( -2\pi \frac{\bar{v} \epsilon}{\sqrt{\bar{N}}\bar{d}}\right) \ln \epsilon -  \exp \left( -2\pi \frac{\bar{v} }{\sqrt{\bar{N}}\bar{d}\epsilon}\right) \ln(1/\epsilon) \right. \\
 & + \left. G_1\left(2\pi \frac{\bar{v} \epsilon}{\sqrt{\bar{N}}\bar{d}}\right)-G_1\left( 2\pi \frac{\bar{v} }{\sqrt{\bar{N}}\bar{d}\epsilon}\right) \right)\\
= & -\frac{\sqrt{\bar{N}}\bar{d}}{2 \pi\bar{v}} \left( \ln \epsilon \left( \exp \left( -2\pi \frac{\bar{v} \epsilon}{\sqrt{\bar{N}}\bar{d}}\right) +  \exp \left( -2\pi \frac{\bar{v} }{\sqrt{\bar{N}}\bar{d}\epsilon}\right) \right) \right. \\
 & + \left. G_1\left(2\pi \frac{\bar{v} \epsilon}{\sqrt{\bar{N}}\bar{d}}\right)-G_1\left( 2\pi \frac{\bar{v} }{\sqrt{\bar{N}}\bar{d}\epsilon}\right) \right)\\
\end{align*}

Now the second integral:
\begin{align*}
\int_{1/ \epsilon}^{\epsilon} \exp \left( -2\pi \frac{\bar{v} y_2}{\sqrt{\bar{N}}\bar{d}}\right) dy_2 
& =  - \frac{\sqrt{\bar{N}}\bar{d}}{2 \pi \bar{v}} \left( \exp \left( -2\pi \frac{\bar{v} \epsilon}{\sqrt{\bar{N}}\bar{d}}\right) -  \exp \left( -2\pi \frac{\bar{v} }{\sqrt{\bar{N}}\bar{d}\epsilon}\right)\right)
\end{align*}

Altogether we have:
\begin{align*}
\int_{1/\epsilon}^{\epsilon} \int_{1}^{\infty} & f\left(\frac{iy_1}{\sqrt{N}},\frac{iy_2}{\sqrt{\bar{N}}}\right) (\ln y_1+\ln y_2)dy_1 dy_2 \\
= &- \frac{\mathbb{N}(N)^{1/2}\mathbb{N}(d)}{4 \pi^2 }\sum_{v \in \mathcal{O}_+}  \frac{a_v}{\mathbb{N}(v)}  \exp\left( -2\pi \frac{v}{\sqrt{N} d}\right) \left( \ln \epsilon \left( \exp \left( -2\pi \frac{\bar{v} \epsilon}{\sqrt{\bar{N}}\bar{d}}\right) +  \exp \left( -2\pi \frac{\bar{v} }{\sqrt{\bar{N}}\bar{d}\epsilon}\right) \right) \right. \\
& + \left. G_1\left(2\pi \frac{\bar{v} \epsilon}{\sqrt{\bar{N}}\bar{d}}\right)-G_1\left( 2\pi \frac{\bar{v} }{\sqrt{\bar{N}}\bar{d}\epsilon}\right) \right) \\
- & \frac{\mathbb{N}(N)^{1/2}\mathbb{N}(d)}{4 \pi^2 } \sum_{v \in \mathcal{O}_+} \frac{a_v}{\mathbb{N}(v)}  G_1\left( 2\pi \frac{v}{\sqrt{N} d}\right) \left( \exp \left( -2\pi \frac{\bar{v} \epsilon}{\sqrt{\bar{N}}\bar{d}}\right) -  \exp \left( -2\pi \frac{\bar{v} }{\sqrt{\bar{N}}\bar{d}\epsilon}\right)\right)
\end{align*}

The first derivative is this whole mess times $(1 - \epsilon_N)$.

We now compute $\Lambda^{(k)}(1,f)$ for general $k$. Recall that:
\begin{equation*}
\begin{split}
\int_{1}^{\infty} &f\left(\frac{iy_1}{\sqrt{N}},\frac{iy_2}{\sqrt{\bar{N}}}\right) (\ln y_1+\ln y_2)^k dy_1\\
\qquad = &\sum_{v \in \mathcal{O}_+} a_v \frac{\sqrt{N}d}{2 \pi v} \exp \left( -2\pi \frac{\bar{v} y_2}{\sqrt{\bar{N}}\bar{d}}\right) \left((\ln y_2)^k \exp\left( -2\pi \frac{v}{\sqrt{N} d}\right) +k \int_1^{\infty} \exp\left( -2\pi \frac{vy_1}{\sqrt{N} d}\right) (\ln y_1+\ln y_2)^{k-1} \frac{dy_1}{y_1}\right)\\
\qquad = &\sum_{v \in \mathcal{O}_+} a_v \frac{\sqrt{N}d}{2 \pi v} \exp\left( -2\pi \frac{v}{\sqrt{N} d}\right) \exp \left( -2\pi \frac{\bar{v} y_2}{\sqrt{\bar{N}}\bar{d}}\right) (\ln y_2)^k \\
\qquad& +  k \sum_{v \in \mathcal{O}_+} a_v \frac{\sqrt{N}d}{2 \pi v} \exp \left(  \frac{-2\pi\bar{v} y_2}{\sqrt{\bar{N}}\bar{d}}\right) \int_1^{\infty} \exp\left( \frac{-2\pi vy_1}{\sqrt{N} d}\right) \left(\sum_{i=0}^{k-1} \binom{k-1}{i} (\ln y_1)^i(\ln y_2)^{k-1-i} \right)\frac{dy_1}{y_1}\\
\qquad = &\sum_{v \in \mathcal{O}_+} a_v \frac{\sqrt{N}d}{2 \pi v} \exp\left( -2\pi \frac{v}{\sqrt{N} d}\right) \exp \left( -2\pi \frac{\bar{v} y_2}{\sqrt{\bar{N}}\bar{d}}\right) (\ln y_2)^k \\
\qquad& +  k \sum_{i=0}^{k-1} \binom{k-1}{i} \sum_{v \in \mathcal{O}_+} a_v \frac{\sqrt{N}d}{2 \pi v} \exp \left(  \frac{-2\pi\bar{v} y_2}{\sqrt{\bar{N}}\bar{d}}\right) (\ln y_2)^{k-1-i} \int_1^{\infty} \exp\left( \frac{-2\pi vy_1}{\sqrt{N} d}\right) (\ln y_1)^i \frac{dy_1}{y_1}\\
\qquad = &\sum_{v \in \mathcal{O}_+} a_v \frac{\sqrt{N}d}{2 \pi v} \exp\left( -2\pi \frac{v}{\sqrt{N} d}\right) \exp \left( -2\pi \frac{\bar{v} y_2}{\sqrt{\bar{N}}\bar{d}}\right) (\ln y_2)^k \\
\qquad& +  k \sum_{i=0}^{k-1} \binom{k-1}{i} \sum_{v \in \mathcal{O}_+} a_v \frac{\sqrt{N}d}{2 \pi v} \exp \left(  \frac{-2\pi\bar{v} y_2}{\sqrt{\bar{N}}\bar{d}}\right) (\ln y_2)^{k-1-i} (i+1)! G_{i+1}\left( \frac{2\pi v}{\sqrt{N} d}\right) 
\end{split}
\end{equation*}

We must now integrate with respect to $y_2$. To do so, we need a little computation lemma:
\begin{lemma}
Let $A$, $B$, and $C$ be positive real numbers and $n$ be a positive integer. Then
\begin{equation*}
\begin{split}
\int_A^B &\exp(-Ct)(\ln t)^n dt \\
= &-\frac{1}{C}\left( \exp(-BC)(\ln B)^n- \exp(-AC)(\ln A)^n + n \sum_{i=0}^{n-1} \binom{n-1}{i} \left( (\ln B)^{n-1-i}G_i(BC)- (\ln A)^{n-1-i}G_i(AC) \right) \right)
\end{split}
\end{equation*}
\end{lemma}
\begin{proof}
Integration by parts, then a change of variables to get $G_i$, which forces a binomial expansion.
\end{proof}

In particular, if $n$ is a positive integer, we have
\begin{equation*}
\begin{split}
\int_{1/\epsilon}^{\epsilon} &\exp \left(  \frac{-2\pi\bar{v} y_2}{\sqrt{\bar{N}}\bar{d}}\right) (\ln y_2)^{n} dy_2\\
= &\frac{\sqrt{\bar{N}}\bar{d}}{2 \pi \bar{v}} \left((\ln \epsilon)^n \left( \exp \left(- \frac{2\pi\bar{v} \epsilon}{\sqrt{\bar{N}}\bar{d}}\right) +(-1)^{n+1} \exp\left(  \frac{-2\pi\bar{v}}{\sqrt{\bar{N}}\bar{d}\epsilon}\right) \right)\right. \\
& \left. \qquad+ n \sum_{i=0}^{n-1} \binom{n-1}{i} (\ln \epsilon)^{n-1-i} \left( G_i\left(\frac{2\pi\bar{v} \epsilon}{\sqrt{\bar{N}}\bar{d}}\right) +(-1)^{n-i}G_i\left(  \frac{2\pi\bar{v}}{\sqrt{\bar{N}}\bar{d}\epsilon}\right) \right) \right)
\end{split}
\end{equation*}

We now have a formula for 
\begin{equation*}
\int_{1/\epsilon}^{\epsilon}\int_{1}^{\infty} f\left(\frac{iy_1}{\sqrt{N}},\frac{iy_2}{\sqrt{\bar{N}}}\right) (\ln y_1+\ln y_2)^k dy_1 dy_2,
\end{equation*}
which we do not put together since it would not be instructive. $\Lambda^{(k)}(1,f)$ is this whole mess times $(1 +(-1)^k \epsilon_N)$.

\bibliographystyle{amsplain}
\bibliography{bibliography}

\end{document}